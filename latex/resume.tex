
\documentclass{resume} % Use the custom resume.cls style

\usepackage[left=0.75in,top=0.6in,right=0.75in,bottom=0.6in]{geometry} % Document margins
\usepackage{hyperref}
\hypersetup{
	colorlinks=true,
	urlcolor=blue,
}

% add tabular command

\newcommand{\btab}[2]{
	\bgroup
	\def\arraystretch{#1}
	\begin{tabular}{#2}
}

\newcommand{\etab}{
	\end{tabular} \smallskip
	\egroup
}

\name{Dylan Johnson} % Your name
\address{3101 S. Taylor St. Apt 745 \\ Little Rock, Arkansas 72204} % Your address
\address{(501)~$\cdot$~547~$\cdot$~1621 \\ dylanatasmsa@gmail.com} % Your phone number and email

\begin{document}

% Work Experience
\begin{rSection}{Experience}

% Emerging Analytics Center
\begin{rSubsection}{Emerging Analytics Center, UALR}{October 2014 - Present}{Software Engineering Intern}{Little Rock, AR}

	\item Worked under Dr. Carolina Cruz-Neira, the inventor of the CAVE system.
	\item Developed Data Visualization solutions for Oculus Rift and CAVE system.
	\item Used Unity 3D C\# for 3D programming and model manipulation.
	\item Used OpenCV (used OpenCvSharp to integrate with C\#) for computer vision applications.
\end{rSubsection}

\begin{rSubsection}{Search User Interface Team, Ancestry.com}{June 2014 - August 2014}{Software Engineering Intern}{Provo, UT}

    \item Worked with a team of engineers to remove and update an old and restrictive codebase.
    \item Used C\#, ASP.NET MVC3, Microsoft SQL Server, Visual Studio 2012, Team Foundatin Server, Windows 7 to develop software.
	\item Used the SCRUM agile software development framework.
\end{rSubsection}

\end{rSection}

\begin{rSection}{Volunteer}

	\begin{rSubsection}{IEEE Virtual Reality Conference}{March 23, 2015 - March 27, 2015}{Student Volunteer}{Arles, France}

		\item Was accepted as a student volunteer for the 2015 IEEE Virtual Reality conference.
		\item I will give 20 hours of volunteer work and recieve conference registration and proceedings.
	\end{rSubsection}

\end{rSection}

% Education Section

\begin{rSection}{Education}

\begin{rSubsection}{University of Arkansas, Little Rock}{August 2014 - Expected Graduation Year: 2017}{B.S. in Computer Science}

	\item Courses: Language Structure, Computer Systems and Assembly Language, Linear Algebra, Operating Systems, Databases, Theory of Computation.
\end{rSubsection}

\begin{rSubsection}{Arkansas School for Mathematics, Sciences, and the Arts}{August 2012 - May 2014}{High School Graduation}

	\item Third place Intel International Science and Engineering Fair project in Materials Engineering at the local level for my research on the optimization of aluminum can camping stoves.
	\item Courses: AP Calculus AB, Calculus 2, Calculus 3 (Vector Calculus), Advanced placement Physics C Mechanics, Advanced Placement Physics C Electricity and Magnetism, Computer Programming 1, Computer Programming 2, Data Structures and Algorithms, Introduction to Web Application Development, Graphics Programming, Discrete Mathematics.
\end{rSubsection}

\end{rSection}

\begin{rSection}{Online}

	\item {\href{https://github.com/dcjohnson}{https://github.com/dcjohnson}}
	\item {\href{https://www.linkedin.com/in/DylanJohnson1}{https://www.linkedin.com/in/DylanJohnson1}}
\end{rSection}

\clearpage

% Personal Projects
\begin{rSection}{Personal Projects}

% CodeBin
\begin{rSubsection}{\href{https://github.com/dcjohnson/CodeBin}{CodeBin}}{December 2014 - Present}{}{}

	\item An online Python development environment.
	\item The backend was build using Python 3.4 and Django 1.7.
	\item The frontend uses Ace for the code editor and Skulpt for the Python interpreter.
	\item Currently the user can create new projects and access them with permalinks, fork projects, browse projects, edit and save projects, and make projects public or private.
\end{rSubsection}

\begin{rSubsection}{Rust}{November 2014 - Present}{}{}
	\item My current projects through which I am learning the Rust Programming Language that is being developed by Mozilla.
	\item {\href{https://github.com/dcjohnson/Rust-Game}{Rust-Game}} My current attempt to build a small multiplayer game in Rust.
	\item {\href{https://github.com/dcjohnson/Little-Rust-Tcp}{Little-Rust-Tcp}} The small TCP Socket library that I am writing for the game.
	\item Both of the mentioned projects are being managed using Cargo.
\end{rSubsection}

\end{rSection}

% Technical Strengths

\begin{rSection}{Technical Strengths}

{\bf General Programming}

\btab{1.1}{ l l }
	Languages: & C, Java, Python, C\#, Rust, JavaScript. \\
	Version Control Systems: & git, Team Foundation Server. \\
	Debuggers: & gdb, Visual Studio 2012 Debugging Software. \\
	Package Managers: & Cargo(Rust), PIP(Python). \\
\etab

{\bf Web Programming}

\btab{1.1}{ l l }
	Server side: & Python, C\#, PHP, and SQL (MySQL, Microsoft SQL Server). \\
	Client side: & JavaScript, CSS, (X)HTML. \\
	Servers: & Apache, Internet Information Services. \\
	Web Frameworks: & Django, ASP.NET MVC3 C\#. \\
	Common Gateway Interface: & Python. \\
\etab

{\bf Development Enviromnents}

\btab{1.1}{ l l }
    IDEs: & Visual Studio 2012, Visual Studio 2013, Monodevelop, Xamarin. \\
	Text Editors: & Vim, Sublime Text, Atom. \\
\etab

{\bf Operating Systems}

\btab{1.1}{ l l }
	Linux: & Debian, Ubuntu, Manjaro, Linux Mint, Slackware. \\
	Windows: & XP, 7, 8. \\
\etab

{\bf Unix System Administration}

\btab{1.1}{ l l }
	Proficient with command line utils: & grep, ssh, vim, nano, tar etc. \\
	Experience with: & configuration files, package managers, compilation, makefiles. \\
\etab

\end{rSection}

\end{document}
